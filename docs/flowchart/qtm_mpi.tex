\documentclass[11pt]{minimal}

\usepackage{tikz}
\usepackage{pgf}

\usepackage[utf8]{inputenc}

\usepackage{times}
%\usepackage{fontspec,xltxtra,xunicode}
%\defaultfontfeatures{Mapping=tex-text}
%\setromanfont[Mapping=tex-text]{Hoefler Text} % Main document font
%\setsansfont[Scale=MatchLowercase,Mapping=tex-text]{Gill Sans} % Font for your name at the top
%\setmonofont[Scale=MatchLowercase]{Andale Mono}
%\usetikzlibrary{shapes,arrows,calc}
\usetikzlibrary{calc,trees,positioning,arrows,chains,shapes.geometric,%
    decorations.pathreplacing,decorations.pathmorphing,shapes,%
    matrix,shapes.symbols}
%%%<
\usepackage{verbatim}
\usepackage{bm}
\usepackage{mystyle}

\usepackage[active,tightpage]{preview}
\PreviewEnvironment{tikzpicture}
%\setlength\PreviewBorder{10pt}%
%%%>

\begin{comment}
:Title: Simple flow chart
:Tags: Diagrams

With PGF/TikZ you can draw flow charts with relative ease. This flow chart from [1]_
outlines an algorithm for identifying the parameters of an autonomous underwater vehicle model. 

Note that relative node
placement has been used to avoid placing nodes explicitly. This feature was
introduced in PGF/TikZ >= 1.09.

.. [1] Bossley, K.; Brown, M. & Harris, C. Neurofuzzy identification of an autonomous underwater vehicle `International Journal of Systems Science`, 1999, 30, 901-913 
\end{comment}


\begin{document}
%\pagestyle{empty}

\tikzset{
    %Define standard arrow tip
    >=stealth',
    %Define style for boxes
    	block/.style={rectangle, fill=brown!10,rounded corners, draw=black, very thick, text width=4cm, minimum height=3em, text centered},
    	block_left/.style={rectangle, fill=brown!10,rounded corners, draw=black, very thick, text width=4cm, minimum height=2em},
        root/.style={rectangle, fill=brown!10,rounded corners, draw=black, very thick, text width=4cm, minimum height=2em},
	base/.style={draw, on chain, on grid, align=center, minimum height=4ex},
    % Define arrow style
	pil/.style={->, thick, shorten <=2pt, shorten >=2pt},
   	 decision/.style={diamond,fill=blue!20,
%	draw=thick, 
	text width = 2cm,
	text badly centered,
	node distance=2.5cm,
	inner sep=2pt,
	rounded corners},
	cloud/.style={ ellipse, fill=pink!10, draw=black, very thick, node distance=6cm, text width = 3cm, text centered,
         minimum height=1.2cm},       
    	line/.style={draw, very thick, -latex'},
	it/.style={}
}

% Define block styles
%\tikzstyle{decision} = [diamond, draw, fill=blue!20, 
%    text width=6em, minimum height = 2em,
%    text badly centered, node distance=3cm, inner sep=0pt]
%\tikzstyle{block} = [rectangle, draw=black, fill=white, %blue!20, 
%    text width=5em, text centered, rounded corners, minimum height=4em]
%\tikzstyle{line} = [draw, -latex']
%\tikzstyle{cloud} = [draw, ellipse,fill=red!20, node distance=3cm,
%    minimum height=2em]
    
\begin{tikzpicture}[
	node distance = 2.cm and 4cm, auto
]
    % Place nodes
    	\node [block, fill = green!10] (init) {Initialize sampling of quantum trajectories, $\bm x,\bm p,\bm r$};
	\node[block,fill = green!10, below of = init, node distance=1.6cm](dist){Distribute quantum trajectories to slave processors};
	\node[block,fill = green!10, below of = dist, node distance = 1.5cm](proc1){traj[$1..N$]};
	\node[block, right of = proc1, node distance = 4.5cm](proc2){traj[$N+1..2N$], $\cdots$ };
	\node[block, right of = proc2, node distance= 4.5cm](procn){traj[$N_{traj}-N+1..N_{traj}$]};
			
        \node [block,fill = green!10, below of=proc1, node distance = 1.5cm] (collect) {Collect averages  over trajectories $\bra x_i \ket, \bra x_ix_j \ket, \cdots $ };

	\node [block_left, fill = green!10, below of=collect](matrix){Call LAPACK to solve $\textbf{A}\bm x=\bm b$, update $U(\bm x,t), \partial_i U(\bm x,t)$};
        \node [block_left, fill = green!10, below of=matrix, text width = 4cm] (fit) {Distribute quantum force to slave processors};
%    \node [cloud, left of=init] (expert) {Coordinates of Nuclei};
   	\node [block_left,fill = green!10, below of=fit] (prop1) {1. Compute classical force\\ 2. Propagate trajectories by $\delta t$};
        \node [block_left, right of=prop1, node distance = 4.5cm] (prop2) {1. Compute classical force\\ 2. Propagate trajectories by $\delta t$};
        \node [block_left, right of=prop2,node distance = 4.5cm] (propn) {1. Compute classical force\\ 2. Propagate trajectories by $\delta t$};


%    \node [block,below of=evaluate](ksorb){Calculate New Electron Density $n(\bm r)$ and Total Energy };
%    \node [block, left of=evaluate, node distance=5.5cm] (update) {update density};
    \node [decision, below of=prop1, node distance=3cm] (decide) {Final time?};
    \node [decision, fill= red, right of=decide, node distance = 3.5cm](stop){STOP};
     \node [it, above=0mm of init](rank0) {\textbf{ROOT} (RANK 0)};
     \node [it, right of = rank0, node distance=4.5cm](rank1) {RANK 1..$N_p$-1};
    \node [it, right of = rank1, node distance=4.5cm]{RANK $N_p$};
     
%    \node [block, below of=decide, node distance=2.5cm] (stop) {Output Quantities};
% A couple of boxes have annotations
%    \node [it, above=0mm of add] {(The main problem in DFT)};

    % Draw edges

%    \path [line] (evaluate) -- (ksorb);
%    \path [line](ksorb)--(decide);
%    \path [line] (decide) -| node [near start] {no} (update);
%    \path [line] (update) |- ($(init)!.5!(identify)$);
  \path [line] (init) -- (dist);
  \path [line] (dist) -- (proc1);
   \path [line] (dist) -| (proc2);
    \path [line] (dist) -| (procn);
    \path [line] (proc1) -- (collect);
    \path [line] (collect) -- (matrix);
    \path [line] (matrix) -- (fit);
    \path [line] (fit) -| (prop2);
    \path [line] (fit) -| (propn);
    \path [line] (fit) -- (prop1);
    \path [line] (proc2) |- (collect);
        \path [line] (procn) |- (collect);
        \path [line] (prop1) -- (decide);
        \path[line]  (decide) edge[bend left=90]  node[near start]{no} (collect);
        \path [line] (decide) -- node {yes}(stop);
%    \path [line,dashed] (expert) -- (init);
%    \path [line,dashed](add)|- (identify);
%    \path [line,dashed] (system) -- (init);
%    \path [line,dashed] (system) |- (evaluate);
%	\path [line, red] (decide) |- ($(init)!.5!(identify)$);

\end{tikzpicture}

\end{document}
