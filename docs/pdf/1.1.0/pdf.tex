\documentclass[11pt]{revtex4}
\usepackage[utf8x]{inputenc}
\usepackage{graphicx}
\usepackage{amssymb}
\usepackage{epstopdf}
\usepackage{mystyle}
\usepackage{amsmath}
\usepackage{bm}

\begin{document}

\title{Pair distribution function}
\author{Bing Gu}

\begin{abstract}

\end{abstract}

\maketitle
\section{Radial distribution function}
To have a theoretical measure of the inter-atom distance distribution at ground state for atomic solids accounting for quantum effects, define radial distribution function for quantum systems in analogy to the formula in statistic mechanics. 
 
\be g(\bm r_1, \bm r_2) = \frac{N(N-1)}{\rho^2} \rho(\bm r_1, \bm r_2) \ee
 
where $\rho = \frac{N}{V}$ is the single particle density and $\rho(\bm r_1, \bm r_2)$ is the joint probability. 
\be \rho(\bm r_1, \bm r_2) = \int d\bm r_3\cdots d\bm r_N~ \rho(\bm r_1,\cdots,\bm r_N) \ee
The one-dimensional radial distribution function can be obtained through averaging of the center of mass and polar angles $\theta$ and $\phi$.  

\be g(r) = \int g(|r_{12}|)\delta(r-|r_{12}|)d|r_{12}| = g(\bm r_1, \bm r_2) = \frac{N(N-1)}{\rho^2} \rho(\bm r_1, \bm r_2) \ee 
\be g(|r_{12}|) = \frac{\int \int \int g(\bm r_1, \bm r_2) d(\frac{\bm r_1+\bm r_2}{2}) d\phi d\cos(\theta)}{\int d(\frac{\bm r_1+\bm r_2}{2}) \int d\phi \int d\cos(\theta)} \ee 

Since the wavefunctions is represented by an ensemble of quantum trajectories, we want an expression that includes terms that can be computed with trajectories.  
%\begin{align} 
% g(r) & = \frac{1}{4\pi \rho N r^2}\frac{\bra \psi_0(\bm r_1, \cdots, \bm r_N)  | \sum_{i,j \neq i}^N \delta(r - |r_{ij}|) | \psi_0(\bm r_1, \cdots, \bm r_N) \ket}{\bra \psi_0(\bm r_1, \cdots, \bm r_N ) | \psi_0(\bm r_1, \cdots, \bm r_N) \ket} \\ 
%	& =   \frac{1}{4\pi \rho N r^2}\sum_{i,j < i}^N \frac{\bra \psi_0(\bm r_1, \cdots, \bm r_N)  | \delta(r - |r_{ij}|) | \psi_0(\bm r_1, \cdots, \bm r_N) \ket}{\bra \psi_0(\bm r_1, \cdots, \bm r_N ) | \psi_0(\bm r_1, \cdots, \bm r_N) \ket}
%\end{align}
%for wavefunction with exchange symmetry, i.e. particles are indistinguishable 
\be g(r)  = \frac{N(N-1)}{\rho^2}\left \bra \frac{ \delta(r - |r_{12}|)}{4\pi V |r_{12}|^2}  \right \ket \ee 
\be g(r) = \frac{N-1}{4\pi \rho} \left \bra \frac{ \delta(r - |r_{12}|)}{|r_{12}|^2}  \right \ket \ee
Here $\bra \dots \ket$ represents quantum ensemble average.  
% \frac{\bra \psi_0(\bm r_1, \cdots, \bm r_N)  | \delta(r - |r_{12}|) | \psi_0(\bm r_1, \cdots, \bm r_N) \ket}{\bra \psi_0(\bm r_1, \cdots, \bm r_N ) | \psi_0(\bm r_1, \cdots, \bm r_N) \ket} \\  
% &= \frac{N}{4\pi \rho r^2} \int d\bm r_1 d\bm r_2 \delta(r - |\bm r_{12}|)\rho(\bm r_1,\bm r_2) 

%We would like to see if the form would give right solution at two limiting cases. The first limiting case is a perfect solid, every atom has a fixed position in space, $\bm R_i$.
%\be g(r) = \sum_{i,j \neq i} \delta(r-|\bm R_{ij}|) \ee 
%the pair distribution function will be peaks at all the possible distance, which is what we expect to see. 

%Another case is non-interacting limit and the ground state density can be approximated with a uniform distribution $\rho(\bm r_1, \bm r_2) = (\frac{1}{V})^2$, the system should behave like ideal gas.  
%The PDF will be  
%\begin{flalign}  
%g(r) & = \frac{N}{\rho}  (\frac{1}{V})^2 \int d\bm r_1 d\bm r_2 \frac{\delta(r - | \bm r_{12}|)}{4\pi r^2} \\ 
% 	& = \frac{N}{\rho V} \\ 
%	& = 1 ~( ideal~gas) \\   
%\end{flalign}

%System in reality should be in the intermediate of this two limiting cases, for small vibrations in the equilibrium position ($|\delta \bm r| << |\bm r|$), a spreading of the infinity peak is likely to show up.
 
%\section{for quantum trajectories }
For numerical difficulty, a Gaussian function would be used to replace the $\delta$ function,  \be \delta(x) = \lim_{\alpha \rightarrow \infty} \sqrt{\frac{2\alpha}{\pi}} e^{-\alpha x^2} \ee 
As and we ignore the exchange effects,  we would use Eq. (1)   
\begin{flalign} 
g(r) & =  \lim_{\sigma \rightarrow \infty} \sum_{\alpha,\beta}^N \bra e^{-\sigma (r-|\bm r_{\alpha \beta} |)^2} \ket  \\
	& =    \lim_{\sigma \rightarrow \infty} \sum_{\alpha,\beta}^N \bra e^{-\sigma (r-|\bm r_{\alpha \beta} |)^2} \ket \\
	& =  \lim_{\sigma \rightarrow \infty}  \sum_{\alpha,\beta}^{N_{atom}} \sum_i^{N_{traj}} w_i e^{-\sigma (r-|\bm r^i_{\alpha \beta} |)^2} \\ 
 \end{flalign}
We can choose a large enough $\sigma$, can compute a bunch of values for $r$.

Also, we can define a range $(R_{min}, R_{max})$, cut it into $N$ pieces, and compute the integration for each interval  
\begin{align}
 \int_{R_j}^{R_{j+1}} dr g(r) & = \sum_{\alpha,\beta}^{N_{atom}} \bra \left (h(R_{j+1}-|\bm r_{\alpha \beta} |)-h(R_j-|\bm r_{\alpha \beta} |)\right) \ket \\
& = \sum_{\alpha,\beta}^{N_{atom}} \sum_i^{N_{traj}} w_i \left (h(R_{j+1}-|\bm r^i_{\alpha \beta} |)-h(R_j-|\bm r^i_{\alpha \beta} |)\right) \\ 
 \end{flalign}

where $h(x)$ is Heaviside step function. 
It is similar to draw a histogram for different intervals, by counting trajectories with $r_{ij}$ fitting that range. 


\end{document}
